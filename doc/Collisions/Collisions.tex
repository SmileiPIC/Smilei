\documentclass[11pt]{article}
\usepackage{amssymb}
\usepackage{amsmath}
\usepackage{graphicx}
\usepackage{fourier}
\usepackage[margin=2cm]{geometry}
\usepackage{fancyvrb,xcolor}
\usepackage{hyperref}
\usepackage{enumerate}
\renewcommand{\labelenumi}{[\theenumi]}

\setlength\parindent{0pt} % no indentation

% code in italics without yellow box
\newcommand{\val}[1]{{\ttfamily \textit{#1}}}

% yellow box for code, inline
\newcommand{\code}[1]{\colorbox{yellow!15}{\ttfamily #1}}

% yellow box for code, full line
\usepackage{listings}
\lstset{backgroundcolor=\color{yellow!15}, basicstyle=\ttfamily, columns=fullflexible, keepspaces=true,escapeinside={�}{�}}


\newlength{\tmp}
\newcommand{\spacing}[1]{\setlength{\tmp}{#1 cm}\addtolength{\tmp}{0.5cm}\hspace*{\tmp}}

\begin{document}

\title{Binary collisions in SMILEI}
\author{F. P\'erez}
\date{March 23, 2015}
\maketitle


Relativistic binary collisions between particles have been implemented in SMILEI with the same scheme as the one developed for the
code CALDER. The following references describe the physics and numerics of this implementation.
\\*

Ref. [1] gives an overview of the technique.\\*
Refs. [2] and [3] give the original technique from which Ref. [1] was developed.\\*
Following Refs. provide additional information.\\*

\begin{enumerate}
\item \href{http://dx.doi.org/10.1063/1.4742167}{F. P\'erez \textit{et al.}, Phys. Plasmas \textbf{19}, 083104 (2012)}
\item \href{http://dx.doi.org/10.1103/PhysRevE.55.4642}{K. Nanbu, Phys. Rev. E \textbf{55}, 4642 (1997)}
\item \href{http://dx.doi.org/10.1006/jcph.1998.6049}{K. Nanbu and S. Yonemura, J. Comput. Phys. \textbf{145}, 639 (1998)}
\item \href{http://dx.doi.org/10.1016/j.jcp.2008.03.043}{Y. Sentoku and A. J. Kemp, J. Comput. Phys. \textbf{227}, 6846 (2008)}
\item \href{http://dx.doi.org/10.1063/1.864744}{Y. T. Lee and R. M. More, Phys. Fluids \textbf{27}, 1273 (1984)}
\item \href{http://dx.doi.org/10.1143/JPSJ.67.4084}{N. E. Frankel, K. C. Hines, and R. L. Dewar, Phys. Rev. A \textbf{20}, 2120 (1979)}
\end{enumerate}

\clearpage

\section{How to add collisions in the input file}

\vspace{1cm}
To have binary collisions in SMILEI, add one or several copies of the following block in the input file:

\begin{lstlisting}
collisions
	species1 = �\textit{species\_name}  \textit{species\_name}  ...�
	species2 = �\textit{species\_name}  \textit{species\_name}  ...�
	coulomb_log = �\textit{ln}$\Lambda$�
end
\end{lstlisting}
\vspace{1cm}

Each \val{species\_name} must be the name (sometimes called "type") of an existing species. This name can be
found in the input file inside any block \code{species ... end} under the keyword \code{species\_type}.\\*

The collisions will occur between (1) all species under the group \code{species1} and (2) all species under the group \code{species2}.
For instance, to have collisions between \code{electrons1} and \code{ions1} , use
\begin{lstlisting}
	species1 = electrons1
	species2 = ions1
\end{lstlisting}
Other example:
\begin{lstlisting}
	species1 = electrons1 electrons2
	species2 = ions
\end{lstlisting}
will collide all electrons with ions.\\*
\textbf{WARNING: this does not make \code{electrons1} collide with \code{electrons2}.}\\*

The two group of species have to be \underline{completely different} OR \underline{exactly equal}.\\*
In other words, if \code{species1} is not equal to \code{species2}, then they cannot have any common species.
If the two groups are exactly equal, we call this situation ``intra-collisions''.

\vspace{1cm}
The value \val{ln$\Lambda$} must be a number.
\begin{itemize}
	\item If \val{ln$\Lambda$} $=0$, the Coulomb logarithm is automatically computed for each collision.
	\item If \val{ln$\Lambda$} $>0$, the Coulomb logarithm is equal to this value.
\end{itemize}


\clearpage


\section{Description of the binary collision scheme}
Collisions are calculated at each timestep.\\*

For each collision block (given in the input file):
\vspace{0.3cm}\\*
\spacing{0} - For each particle cluster:
\vspace{0.3cm}\\*
\spacing{1} - If \textit{intra-collisions}
\vspace{0.3cm}\\*
\spacing{2} Create one array containing the indices pointing to all particles of the species group.\\*
\spacing{2} Shuffle the array.\\*
\spacing{2} Split the array in two halves.
\vspace{0.3cm}\\*
\spacing{1} - Otherwise
\vspace{0.3cm}\\*
\spacing{2} Create two arrays containing the indices pointing to all particles of each species group.\\*
\spacing{2} Shuffle the largest array. The other array is not shuffled.\\*\\*
\vspace*{0.4cm}
\spacing{1}=> The two resulting arrays represent pairs of particles (see algorithm in Ref. [3]).\\*
\spacing{1} - Calculate a few intermediate quantities:
\vspace{0.3cm}\\*
\spacing{2} Particle density $n_1$ of group 1.\\*
\spacing{2} Particle density $n_2$  of group 2.\\*
\spacing{2} ``crossed'' particle density $n_{12}$ (see Ref. [1]).\\*
\spacing{2} Other constants.
\vspace{0.3cm}\\*
\spacing{1} - For each pair of particles:
\vspace{0.3cm}\\*
\spacing{2} - Calculate the momenta in the center-of-mass (COM) frame.\\*
\spacing{2} - Calculate the coulomb log if requested (see Ref. [1]).\\*
\spacing{2} - Calculate the collision parameter $s$ and its correction at low temperature (see Ref. [1]).\\*
\spacing{2} - Pick the deflection angle (see Ref. [2]).\\*
\spacing{2} - Deflect particles in the COM frame and go back to the laboratory frame.\\*

 
\clearpage

\section{Test cases}

\subsection{Beam relaxation}

An electron beam with narrow energy spread enters an ion background with $T_i=10$ eV.\\*
The ions are of very small mass $m_i=10 m_e$ to speed-up the calculation.\\*
Only e-i collisions are calculated.\\*
The beam gets strong isotropization => the average velocity relaxes to zero.\\*

Three figures show the time-evolution of the longitudinal $\left<v_\|\right>$ and transverse velocity $\sqrt{\left<v_\perp^2\right>}$
\begin{itemize}
\item Figure \ref{beam1}: initial velocity = 0.05, ion charge = 1
\item Figure \ref{beam2} : initial velocity = 0.01, ion charge = 1
\item Figure \ref{beam3} : initial velocity = 0.01, ion charge = 3
\end{itemize}
Each of these figures show 3 different blue and red curves which correspond to different ratios of particle weights: 0.1, 1, and 10.\\*

\begin{figure}[h]
\centering
\includegraphics[width=12cm]{beam_relaxation123}
\caption{Relaxation of an electron beam. Initial velocity = 0.05, ion charge = 1.}
\label{beam1}
\end{figure}
\begin{figure}[h]
\centering
\includegraphics[width=12cm]{beam_relaxation456}
\caption{Relaxation of an electron beam. Initial velocity = 0.01, ion charge = 1.}
\label{beam2}
\end{figure}
\begin{figure}[h]
\centering
\includegraphics[width=12cm]{beam_relaxation789}
\caption{Relaxation of an electron beam. Initial velocity = 0.01, ion charge = 3.}
\label{beam3}
\end{figure}


The black lines correspond to the theoretical rates taken from the NRL formulary:\\*
\begin{equation*}
\nu_\| = -\left(1+\frac{m_e}{m_i}\right)\nu_0 \quad\textrm{and}\quad \nu_\perp = 2\;\nu_0 \quad\textrm{where}\quad \nu_0=\frac{e^4\,Z^{\star 2}\,n_i\,\ln\Lambda } { 4 \pi \epsilon_0^2 \,m_e^2\,v_e^3 }
\end{equation*}

\textbf{The distribution is quickly non-Maxwellian so that theory is valid only at the beginning.\\*}

\clearpage
\subsection{Thermalization}
A population of electrons has a different temperature from that of the ion population.\\*
Through e-i collisions, the two temperatures become equal.\\*
The ions are of very small mass $m_i=10 m_e$ to speed-up the calculation.\\*
Three cases are simulated, corresponding to different ratios of weights : 0.2, 1 and 5.\\*
They are plotted in Figure \ref{thermalization}.\\*

\begin{figure}[h]
\centering
\includegraphics[width=9cm]{thermalisation_ei123}
\caption{Thermalization between two species.}
\label{thermalization}
\end{figure}

The black lines correspond to the theoretical rates taken from the NRL formulary:\\*
\begin{equation*}
\nu_\epsilon=\frac{e^4\,Z^{\star 2} \sqrt{m_em_i}\,n_i\,\ln\Lambda } { 8 \epsilon_0^2 \,\left(m_eT_e+m_iT_i\right)^{3/2} }
\end{equation*}

\clearpage

\subsection{Temperature isotropization {\color{red} \danger temporary modification needed in Species.cpp}}
Electrons have a longitudinal temperature different from their transverse temperature.\\*
They collide only with themselves (intra-collisions) and the anisotropy disappears as shown in Figure \ref{temperature_isotropization}.\\*

\begin{figure}[h]
\centering
\includegraphics[width=8cm]{temperature_isotropization1}
\caption{Temperature isotropization of an electron population.}
\label{temperature_isotropization}
\end{figure}

The black lines correspond to the theoretical rates taken from the NRL formulary:\\*
\begin{equation*}
\nu_T=\frac{e^4 \,n_e\,\ln\Lambda } { 8\pi^{3/2} \epsilon_0^2 \,m_e^{1/2}T_\|^{3/2} } A^{-2} \left[-3+(3-A)\frac{\rm{arctanh}(\sqrt{A})}{\sqrt{A}}\right]
\quad \rm{where}\quad A=1-\frac{T_\perp}{T_\|}
\end{equation*}


\subsection{Maxwellianization {\color{red} \danger temporary modification needed in Species.cpp}}
Electrons start with zero temperature along y and z. Their velocity distribution along x is a boxcar.\\*
They collide only with themselves and the boxcar becomes a maxwellian as shown in Figure \ref{maxwellianization}.

\begin{figure}[h]
\centering
\includegraphics[width=8.5cm]{Maxwellianization1}
\caption{Maxwellianization of an electron population. Each blue curve is the distribution at a given time. The red curve is an example of a gaussian function.}
\label{maxwellianization}
\end{figure}


\subsection{Stopping power}
Test electrons (very low density) collide with background electrons of density 10 $n_c$ and $T_e=$ 5 keV.\\*
Depending on their initial velocity, they are slowed down at different rates, as shown in Figure \ref{stoppingpower}.\\*

\begin{figure}[h]
\centering
\includegraphics[width=8.5cm]{Stopping_power123}
\parbox{14cm}{\caption{Stopping power of test electrons into a background electron population. Each point is one simulation.
The black line is Frankel's theory (Ref. [6]).}}
\label{stoppingpower}
\end{figure}



\subsection{Conductivity {\color{red} \danger temporary modification needed in Smilei.cpp}}
Solid-density Cu is simulated at different temperatures (e-i equilibrium) with only e-i collisions.\\*
An electric field of $E=3.2$ GV/m (0.001 in code units) is applied.\\*
The electron velocity increases until a limit value $v_f$.\\*
The resulting conductivity $\sigma=en_ev_f/E$ is compared in Figure \ref{conductivity} to the models in Refs. [5] and [1].

\begin{figure}[h]
\centering
\includegraphics[width=9cm]{conductivity}
\caption{Conductivity of colid-density copper. Each point is one simulation.}
\label{conductivity}
\end{figure}


\end{document}


