\documentclass[11pt]{article}
\usepackage{graphicx}
\usepackage[margin=2cm]{geometry}
\usepackage{fancyvrb,xcolor}
\usepackage[urlcolor=blue,linkcolor=black,colorlinks=true]{hyperref}
\urlstyle{rm}
\usepackage{wrapfig}
\usepackage{tabularx}

\setlength\parindent{0pt}

\newcommand{\code}[1]{\colorbox{yellow!15}{\ttfamily #1}}
\newcommand{\val}[1]{{\ttfamily \textit{#1}}}
\newcommand{\blue}[1]{{\color{blue} #1}}

\usepackage{listings}
\lstset{backgroundcolor=\color{yellow!15}, basicstyle=\ttfamily, columns=fullflexible, keepspaces=true,escapeinside={�}{�}}

\begin{document}

\title{Profiles in SMILEI}
\author{F. P\'erez}
\maketitle

\vfill

Several quantities require the input of a spatial and/or temporal profile:
\begin{itemize}
\item Species charge: to define the average charge at each point.
\item Species density: to define either charge density or number density at each point.
\item Species average velocity: to define an offset of the velocity distribution at each point.
\item Species temperature: to define the width of the velocity distribution at each point 
	(valid only for distributions which require a temperature value).
\item Number of particles per cell.
\item External field: to define an initial external field at each point.
\item Laser: to define a temporal and/or spatial laser enveloppe.
\end{itemize}

\vfill

\setcounter{tocdepth}{2}
\tableofcontents
\vfill

\clearpage


\section{Species profiles\label{sec:species}}

For each species, it is possible to define spatial profiles that will determine the initial distribution
of the particles. There are many ways to define a profile.
\vspace{2cm}

\subsection{First technique: no profile}
You may decide to have a constant profile: a constant value over the whole box.
In this case, it is very easy: you do not need to define a profile.
\vspace{1cm}

Charge:
\begin{lstlisting}
Species( ... , charge = -3., ... )
\end{lstlisting}
defines a species with charge $Z^\star=3$ for all its particles.
\vspace{1cm}

Density: You can choose \code{nb\_density} (\textit{number density}) or \code{charge\_density}
\begin{lstlisting}
Species( ... , nb_density = 10., ... )
\end{lstlisting}
defines a species with density $10\,n_c$ over the whole box ($n_c$ is the critical density
 of light which wavelength is equal to the normalization length).
\vspace{1cm}

Drift velocity:
\begin{lstlisting}
Species( ... , mean_velocity = [0.05, 0., 0.], ... )
\end{lstlisting}
defines a species with drift velocity $v_x = 0.05\,c$ over the whole box ($c$ is the speed of light).
\vspace{1cm}

Temperature:
\begin{lstlisting}
Species( ... , initMomentum_type = "maxwell-juettner", temperature = [1e-5], ... )
\end{lstlisting}
defines a species with a Maxwell-J\"uttner distribution of temperature $T = 10^{-5}\,m_ec^2$ over the whole box.
Note that the temperature may be anisotropic: \code{temperature=[1e-5, 2e-5, 2e-5]}.
\vspace{1cm}

Number of macro-particles per cell:
\begin{lstlisting}
Species( ... , n_part_per_cell = 10., ... )
\end{lstlisting}
\vfill
 \clearpage
  
 \subsection{Second technique: \emph{python} profile}
 Any \emph{python} function can be a profile. You must have basic \emph{python} knowledge to build these functions. Examples:
\begin{lstlisting}
def f(x):
	if x<1.: return 0.
	else: return 1.
\end{lstlisting}
\begin{lstlisting}
def f(x,y):    # two variables for 2D simulation
	import math
	twoPI = 2.* math.pi
	return math.cos(  twoPI * x/3.2 )
\end{lstlisting}
\begin{lstlisting}
f = lambda x: x**2 - 1
\end{lstlisting}
\vspace{1cm}

Once the function is created, you have to include it in the species definition:
\begin{lstlisting}
Species( ... , charge = f, ... )
\end{lstlisting}
\begin{lstlisting}
Species( ... , nb_density = f, ... )
\end{lstlisting}
\begin{lstlisting}
Species( ... , mean_velocity = [f, f, f], ... )
\end{lstlisting}
\begin{lstlisting}
Species( ... , temperature = [f, f, f],	... )
\end{lstlisting}
\begin{lstlisting}
Species( ... , n_part_per_cell = f, ... )
\end{lstlisting}
and so on ...
\vfill

\subsection{Third technique: built-in \emph{python} functions\label{sec:pyprofiles}}
SMILEI provides some \emph{python} functions to help you build your profiles.
\vspace{1cm}

To define a \underline{constant profile}:
\begin{lstlisting}
�constant(value, xvacuum=0., yvacuum=0.)�
\end{lstlisting}
\code{value} defines the magnitude.\\
\code{xvacuum} and \code{yvacuum} are vacuum regions before the start of the profile.
\vspace{1cm}

To define a \underline{trapezoidal profile}:
\begin{lstlisting}
�trapezoidal(max,�
            �xvacuum=0., xplateau=None, xslope1=0., xslope2=0.,�
            �yvacuum=0., yplateau=None, yslope1=0., yslope2=0. )�
\end{lstlisting}
\code{max} is the maximum value, \code{xvacuum} is the empty length before the ramp up,
\code{xplateau} is the length of the plateau (default is {\ttfamily sim\_length - xvacuum}),
\code{xslope1} is the length of the ramp up and \code{xslope2} is the length of the ramp down.
The other arguments are the same, but for a 2D simulation.\\
\vspace{1cm}

To define a \underline{gaussian profile}:
\begin{lstlisting}
�gaussian(max,�
         �xvacuum=0., xlength=None, xfwhm=None, xcenter=None, xorder=2,�
         �yvacuum=0., ylength=None, yfwhm=None, ycenter=None, yorder=2 )�
\end{lstlisting}
\code{max} is the maximum value, \code{xvacuum} is the empty length before starting the profile,
\code{xlength} is the length of the profile (default is {\ttfamily sim\_length - xvacuum}),
\code{xfwhm} is the gaussian FWHM (default is {\ttfamily xlength/3.}),
\code{xcenter} is the gaussian center position (default is in the middle of {\ttfamily xlength}
and \code{xorder} is the order of the gaussian.
The other arguments are the same, but for a 2D simulation. If \code{yorder==0}, then the profile is constant over $y$.\\
\vspace{1cm}

To define a \underline{polygonal profile}:
\begin{lstlisting}
polygonal( xpoints=[], xvalues=[] )
\end{lstlisting}
where \code{xpoints} is a list defining the position of points, and \code{xvalues} is
a list defining the values of the profile at each point.\\
\clearpage

To define a \underline{cosine profile}:
\begin{lstlisting}
�cosine( base, amplitude=1.,
                       xvacuum=0., xlength=None, phi=0., xnumber=1 )
\end{lstlisting}
\code{base} is an offset of the profile value, \code{amplitude} is the amplitude of the cosine,
\code{xvacuum} is the empty length before starting the profile,
\code{xlength} is the length of the profile (default is {\ttfamily sim\_length - xvacuum}),
\code{phi} is the phase offset
and \code{xnumber} is the number of periods within \code{xlength}.\\

\vspace{1cm}
\textbf{Example:}
\begin{lstlisting}
Species( ... ,
	density = gaussian(10., xfwhm=0.3, xcenter=0.8),
	... )
\end{lstlisting}
\vfill

\textbf{Illustration of the built-in profiles}
{
\center
\includegraphics[width=17cm]{pythonprofiles.pdf} 
}
\vfill
\clearpage



\section{External field profiles\label{sec:extfields}}

External field profiles work almost the same way as species profiles.

\vspace{0.7cm}

\subsection{First technique: no profile}
\begin{lstlisting}
ExtField(
	field = "�\val{any\_field}�",
	profile = �\val{magnitude}�
)
\end{lstlisting}
The \code{magnitude} keyword determines a constant value of the chosen external field.

\vspace{1cm}

\subsection{Second technique: \emph{python} function or \emph{python} built-in profile}
\begin{lstlisting}
ExtField(
	field = "�\val{any\_field}�",
	profile = �\val{myProfile}�
)
\end{lstlisting}
where \val{myProfile} is a \emph{python} function or a built-in \emph{python} profile.

All the built-in profiles from section \ref{sec:pyprofiles} are available here.


\clearpage

\section{Laser profiles\label{sec:laser}}

Anybody ?


\end{document}


