\documentclass[11pt]{article}
\usepackage{graphicx}
\usepackage[margin=2cm]{geometry}
\usepackage{fancyvrb,xcolor}
\usepackage[urlcolor=blue,linkcolor=black,colorlinks=true]{hyperref}
\usepackage{wrapfig}

\setlength\parindent{0pt}

\newcommand{\code}[1]{\colorbox{yellow!15}{\ttfamily #1}}
\newcommand{\val}[1]{{\ttfamily \textit{#1}}}

\usepackage{listings}
\lstset{backgroundcolor=\color{yellow!15}, basicstyle=\ttfamily, columns=fullflexible, keepspaces=true,escapeinside={�}{�}}

\begin{document}

\title{The particle diagnostic in SMILEI}
\author{F. P\'erez}
\date{March 19, 2015}
\maketitle

The \textit{particle diagnostic} collects data from the macro-particles and processes them during runtime.
It does \underline{not} provide information on individual particles: instead, it produces \underline{averaged quantities}
like the particle density, currents, etc.
\\*
\\*
The data may be collected from one or several particle species.
\\*
\\*
The data is discretized inside a ``grid'' chosen by the user. This grid may be of any dimension.
\\*
Examples:
\begin{itemize}
\item 1-dimensional grid along the position $x$ (gives density variation along $x$)
\item 2-dimensional grid along positions $x$ and $y$ (gives density map)
\item 1-dimensional grid along the velocity $v_x$ (gives the velocity distribution)
\item 2-dimensional grid along position $x$ and momentum $p_x$ (gives the phase-space)
\item 1-dimensional grid along the kinetic energy $E_\mathrm{kin}$ (gives the energy distribution)
\item 3-dimensional grid along $x$, $y$ and $E_\mathrm{kin}$ (gives the density map for several energies)
\item 1-dimensional grid along the charge $Z^\star$ (gives the charge distribution)
\end{itemize}
Each dimension of the grid is called ``axis".
\\*
\\*
The user may choose to average the data over several time-steps.
\\*
\setcounter{tocdepth}{1}
\tableofcontents

\clearpage

\section{How to add a particle diagnostic in the input file}
In the input file, you can add a particle diagnostic with the following syntax.

\begin{lstlisting}
diagnostic particles
	output = �\textit{output}�
	every = �\textit{every}�
	time_average = �\textit{time\_average}�
	species = �\textit{species1 species2 ...}�
	axis = �\textit{type min max nsteps}� [logscale] [edge_inclusive]
	axis = �\textit{type min max nsteps}� [logscale] [edge_inclusive]
	...
end
\end{lstlisting}

\begin{itemize}
\item \val{output} must be set to \code{density}, \code{charge\_density}, \code{current\_density\_x}, \code{current\_density\_y},
	\code{current\_density\_z}, \code{p\_density}, \code{px\_density}, \code{py\_density} or \code{pz\_density}
	It determines the data that is summed in each cell of the grid.
	In the case of \code{density}, the \textit{weights} are summed.
	In the case of \code{charge\_density}, the \textit{weights}$\times$\textit{charge} are summed.
	In the case of \code{current\_density\_x}, the \textit{weights}$\times$\textit{charge}$\times v_x$ are summed (etc.).
	In the case of \code{p\_density}, the \textit{weights}$\times p$ are summed (etc.)\\*
\item \val{every} must be a positive integer. It is the number of time-steps between each output.\\*
\item \val{time\_average} must be a positive integer. It is the number of time-steps during which the data is averaged before output.\\*
\item \val{species1 species2 ...}  must be one or several species.
	Species are recognized by their parameter \code{species\_type} that is included
	in each group \code{species ... end}\\*
\item \code{axis} is an argument that describes one axis of the grid.\\
	Syntax: \code{axis = \textit{type min max nsteps} [logscale] [edge\_inclusive]}\\
	\val{type} can be \code{x}, \code{y}, \code{z}, \code{px}, \code{py}, \code{pz}, \code{p}, 
	 \code{gamma}, \code{ekin}, \code{vx}, \code{vy}, \code{vz}, \code{v} or \code{charge}.\\
	The axis is discretized for \val{type} from \val{min} to \val{max} in \val{nsteps} bins.\\
	The optional keyword \code{logscale} sets the axis scale to logarithmic instead of linear.\\
	The optional keyword \code{edge\_inclusive} includes the particles outside the range [\val{min},  \val{max}]
	into the extrema bins.\\*\\*
\end{itemize}
There may be as many \code{axis} arguments as wanted in one \code{diagnostic particles} block.
\\*\\*\\*
There may be as many \code{diagnostic particles} blocks as wanted in the input file.

\clearpage
In the input file, you may add the following block as a reminder.

\begin{lstlisting}
# DIAGNOSTICS ON PARTICLES - project the particles on a N-D arbitrary grid
# ------------------------------------------------------------------------
# output = density, charge_density, current_density_[xyz] or p[xyz]_density
#              => parameter that describes what quantity is obtained 
# every        => integer > 0 : number of time-steps between each output
# time_average => integer > 0 : number of time-steps to average
# species      => list of one or several species whose data will be used
# axis   = _type_ _min_ _max_ _nsteps_ [logscale] [edge_inclusive]
#              => _type_ can be one of the following:
#                  x, y, z, px, py, pz, p, gamma, ekin, vx, vy, vz, v or charge
#              => the data is discretized for _type_ between _min_ and
#                  _max_, in _nsteps_ bins
#              => the optional [logscale] sets the scale to logarithmic
#              => the optional [edge_inclusive] forces the particles
#                  outside (_min_,_max_) to be counted in the extrema bins
#   example : axis = x 0 1 30
#   example : axis = px -1 1 100 
# >>>> MANY AXES CAN BE ADDED IN A SINGLE DIAGNOSTIC <<<<
\end{lstlisting}


\clearpage
\section{Examples}

Variation of the density of species \code{electron1} from $x=0$ to 1, every 5 time-steps, without time-averaging.
\begin{lstlisting}
diagnostic particles
	output = density
	every = 5
	time_average = 1
	species = electron1
	axis = x    0    1    30
end
\end{lstlisting}

Density map from $x=0$ to 1, $y=0$ to 1.
\begin{lstlisting}
diagnostic particles
	output = density
	every = 5
	time_average = 1
	species = electron1
	axis = x    0    1    30
	axis = y    0    1    30
end
\end{lstlisting}

Velocity distribution from $v_x = -0.1$ to $0.1$.
\begin{lstlisting}
diagnostic particles
	output = density
	every = 5
	time_average = 1
	species = electron1
	axis = vx    -0.1    0.1    100
end
\end{lstlisting}

Phase space from $x=0$ to 1 and from $px=-1$ to 1.
\begin{lstlisting}
diagnostic particles
	output = density
	every = 5
	time_average = 1
	species = electron1
	axis = x    0    1    30
	axis = px    -1    1    100
end
\end{lstlisting}

Energy distribution from 0.01 to 1 MeV in logarithmic scale.
Note that the input units are $m_ec^2 \sim 0.5$ MeV.
\begin{lstlisting}
diagnostic particles
	output = density
	every = 5
	time_average = 1
	species = electron1
	axis = ekin    0.02    2   100  logscale
end
\end{lstlisting}
\clearpage

$x$-$y$ density maps for three bands of energy: $[0,1]$, $[1,2]$, $[2,\infty]$.
Note the use of \code{edge\_inclusive} to reach energies up to $\infty$.
\begin{lstlisting}
diagnostic particles
	output = density
	every = 5
	time_average = 1
	species = electron1
	axis = x    0    1    30
	axis = y    0    1    30
	axis = ekin    0   6   3  edge_inclusive
end
\end{lstlisting}

Charge distribution from $Z^\star =0$ to 10.
\begin{lstlisting}
diagnostic particles
	output = density
	every = 5
	time_average = 1
	species = electron1
	axis = charge    -0.5   10.5   11
end
\end{lstlisting}




\clearpage
\section{How to view and post-process a particle diagnostic}

Each diagnostic produces one file like \textit{ParticleDiagnostic0.h5}, \textit{ParticleDiagnostic1.h5}, etc. They
are numbered the same way as the order of appearance in the input file.
\\*

A python script \textit{Diagnostics.py} is provided to view or extract data from these files.
To run this script, you will need \textit{python2.7} with the following packages: numpy, matplotlib, pylab, h5py.
\\*

First, run python and include the script: \code{python -i scripts/Diagnostics.py}\\*
Alternately, include this file into your own script: \code{execfile("scripts/Diagnostics.py")}

\subsection{To prepare the data}
\begin{lstlisting}
ParticleDiagnostic(results_path, diagNumber=None, timesteps=None, slice=None,
                    units="code", data_log=False)
\end{lstlisting}

\begin{itemize}
\item \code{results\_path} = \code{\_string\_}\\*
	Path to the directory where the outputs are stored.\\*
	(Also, this has to contain one and only one input file *.in)

\item \code{diagNumber} = \code{\_int\_} or  \code{\_string\_} (optional)\\*
	If not given, then a list of available diagnostics is printed.\\*
	If \code{\_int\_}: number of the diagnostic. The first diagnostic has number 0.\\*
	If \code{\_string\_}: operation between several diagnostics. See section \ref{operations}.

\item \code{timesteps} =  \code{\_int\_ }          (optional)\\*
	\code{timesteps} = \code{[\_int\_, \_int\_]}   (optional)\\*
	If omitted, all timesteps are used.\\*
	If one number  given, the nearest timestep available is used.\\*
	If two numbers given, all the timesteps in between are used.

\item \code{slice} = \code{\{ \val{axis} : "all", ... \}}                 (optional)\\*
	\code{slice} = \code{\{ \val{axis} : \_double\_, ... \}}              (optional)\\*
	\code{slice} = \code{\{ \val{axis} : [\_double\_, \_double\_], ... \}}  (optional)\\*
	This parameter is used to reduce the number of dimensions of the array.\\*
	\val{axis} must be \code{"x"}, \code{"y"}, \code{"z"}, \code{"px"}, \code{"py"}, \code{"pz"}, \code{"p"},
	 \code{"gamma"}, \code{"ekin"}, \code{"vx"}, \code{"vy"}, \code{"vz"}, \code{"v"} or \code{"charge"}.\\*
	 Any axis of the same name will be removed with the following technique:\\*
	- If the value is \code{"all"}, then a sum is performed over all the axis.\\*
	- If the value is \code{\_double\_}, then only the bin closest to the value is kept.\\*
	- If the value is \code{[\_double\_,\_double\_]}, then a sum is performed between the two values.\\*
	Example: \code{\{"x":[4,5]\}} will sum all the data for x in the range [4,5].

\item \code{units} = \code{"nice"}    (optional)\\*
	If \code{"nice"} is chosen, then units are converted into usual units.\\*
	Distances in microns, density in cm$^{-3}$, energy in MeV.

\item \code{data\_log} = \code{True} or (\code{False})       (optional)\\*
	If True, then $\log_{10}$ is applied to the output array.
\end{itemize}


\subsection{To obtain the data as an array:}
\begin{lstlisting}
ParticleDiagnostic( ... ).getData()
\end{lstlisting}
This method returns only a list of the data arrays (for each timestep requested).

\begin{lstlisting}
ParticleDiagnostic( ... ).get()
\end{lstlisting}
This method returns the results as a python dictionary:
\begin{lstlisting}
{"data":�\val{data\_array}�, "times":�\val{times\_array}�, �\val{axis1}�:�\val{axis1\_array}�, �\val{axis2}�:�\val{axis2\_array}�, ...}
\end{lstlisting}
where:
\begin{itemize}
\item \val{data\_array} is the result of \code{getData()}.
\item \val{times\_array} is a list of the requested timesteps.
\item \val{axis1}, \val{axis2} (...) are the names of the axes (for example \code{"x"} or \code{"px"}).
\item \val{axis1\_array}, \val{axis2\_array} (...) are the locations of the axes bins.\\*
\end{itemize}

\textbf{Example:}
\begin{lstlisting}
>>> result = ParticleDiagnostic("path/to/my/results", diagNumber=3,
    slice={"ekin":[1,10]}, timesteps=1000).get()
\end{lstlisting}
This will take the diagnostic \#3 for the timestep nearest to 1000, and sum for all energies between 1 and 10.
The results are stored in the variable \code{result}.\\*
The data can be accessed with \code{result["data"]}.\\*
If one of the axes is \val{x}, you can access the locations of the bins with \code{result["x"]}.\\* \\*

\subsection{To plot the data}
\begin{lstlisting}
ParticleDiagnostic( ... , figure=1, data_min=None, data_max=None,
                          xmin=None, xmax=None, ymin=None, ymax=None ).plot()
\end{lstlisting}

\begin{itemize}
\item \code{figure} = \code{\_int\_}       (optional)\\*
	The figure number that is passed to matplotlib.\\*
	If absent, figure 1 is used.

\item \code{data\_min} = \code{\_double\_}    (optional)\\*
	\code{data\_max} = \code{\_double\_}    (optional)\\*
	If present, output is rescaled before plotting.

\item \code{xmin} = \code{\_double\_}    (optional)\\*
	\code{xmax} = \code{\_double\_}    (optional)\\*
	\code{ymin} = \code{\_double\_}    (optional)\\*
	\code{ymax} = \code{\_double\_}    (optional)\\*
	If present, axes are rescaled before plotting.
\end{itemize}

\textbf{Example:}
\begin{lstlisting}
>>> ParticleDiagnostic("path/to/my/results", diagNumber=1,
                       figure=1,data_min=0, data_max=1e14 ).plot()
\end{lstlisting}
This will take the diagnostic \#1, and sum the grid over all the $y$ axis. Then it will plot the resulting array in figure 1 from 0 to 3e14.
\clearpage

\subsection{To simultaneously plot multiple diagnostics in the same figure}
\begin{lstlisting}
multiPlot(diag1, diag2, ... , figure=1, shape=None)
\end{lstlisting}

\begin{itemize}
\item \code{diag1} = diagnostic prepared by \code{ParticleDiagnostic(...)}\\*
	\code{diag2} = diagnostic prepared by \code{ParticleDiagnostic(...)}\\*
	...

\item \code{figure} = \code{\_int\_}       (optional)\\*
	The figure number that is passed to matplotlib.\\*
	If absent, figure 1 is used.
	
\item \code{shape} = \code{[\_int\_ , \_int\_]}       (optional)\\*
	The arrangement of plots inside the figure. For instance, \code{[2, 1]} makes two plots stacked vertically, and \code{[1, 2]} makes two plots stacked horizontally.\\*
	If absent, stacks plots vertically.
\end{itemize}

\textbf{Example:}
\begin{lstlisting}
>>> A = ParticleDiagnostic("path/to/my/results", diagNumber=0)
>>> B = ParticleDiagnostic("path/to/my/results", diagNumber=1)
>>> multiPlot( A, B, figure=1 )
\end{lstlisting}

This will plot the diagnostics \#0 and \#1 on the same figure, and make an animation for all available timesteps.


\subsection{Advanced plotting options}
In addition to \code{figure}, \code{data\_min}, \code{data\_max}, \code{xmin}, \code{xmax}, \code{ymin} and \code{ymax}, there are many
more optional arguments for the function \code{ParticleDiagnostic(...)}. They are directly passed to the \textit{matplotlib} package.
\begin{itemize}
\item Options for the figure: \code{figsize},\code{dpi},\code{facecolor},\code{edgecolor}\\
	Please refer to \url{http://matplotlib.org/api/pyplot_api.html#matplotlib.pyplot.figure}
\item Options for the lines: \code{color},\code{dashes},\code{drawstyle},\code{fillstyle},\code{label},\code{linestyle},\code{linewidth},\\\code{marker},\code{markeredgecolor},\code{markeredgewidth},\code{markerfacecolor},	\code{markerfacecoloralt},\\
	\code{markersize},\code{markevery},	\code{visible},\code{zorder}\\
	Please refer to \url{http://matplotlib.org/api/pyplot_api.html#matplotlib.pyplot.plot}
\item Options for the image: \code{cmap},\code{aspect},\code{interpolation}\\
	Please refer to \url{http://matplotlib.org/api/pyplot_api.html#matplotlib.pyplot.imshow}
\item Options for the colorbar: \code{cbaspect}, \code{orientation},\code{fraction},\code{pad},\code{shrink},\code{anchor},\code{panchor},\\
	\code{extend},\code{extendfrac},\code{extendrect},\code{spacing},\code{ticks},\code{format},\code{drawedges}\\
	Please refer to \url{http://matplotlib.org/api/pyplot_api.html#matplotlib.pyplot.colorbar}
\end{itemize}
\textbf{Example:}\\
To choose a gray colormap of the image, use \code{cmap="gray"}.
\begin{lstlisting}
>>> A = ParticleDiagnostic("path/to/my/results", diagNumber=0, figure=1, cmap="gray")
\end{lstlisting}
Many colormaps are available from the \textit{matplotlib} package. With \code{cmap=""}, you will get a list of available colormaps.


\subsection{Updating the plotting options}

You can change the plotting parameters using the \code{A.plot()} or the \code{A.set()} functions.\\
\textbf{Example:}
\begin{lstlisting}
>>> A = ParticleDiagnostic("path/to/my/results", diagNumber=0, figure=1, data_max=1)
>>> A.plot( figure=2 )
>>> A.set( data_max=2 )
>>> A.plot()
\end{lstlisting}


\subsection{\label{operations}Operations between several diagnostics}
Sometimes, you have two (or more) diagnostics and you want to combine them with some operation. For example,
imagine that you have one diagnostic with \code{output=px\_density} and another one with \code{output=density}.
You may want to divide the first by the second in order to obtain the average $p_x$.\\*
This is directly possible here by using the argument \code{diagNumber}: you simply need to define this argument
as a string containing an operation of your choice, and use a pound sign (\#) to indicate each diagnostic. For example,
the string \code{"\#1 / \#0"} will achieve the division between diagnostics \#1 and \#0.\\*
This feature is very useful as you can make any type of operation, as long as the two diagnostics have the same axes
and the same timesteps.\\
\textbf{Example:}
\begin{lstlisting}
>>> A = ParticleDiagnostic("path/to/my/results", diagNumber="(#1 + #2) / #0").plot()
\end{lstlisting}


\clearpage
\section{Bonus: view fields and scalars as well}

SMILEI provides maps of the fields $\overrightarrow E$, $\overrightarrow B$, $\rho$ (etc.) controlled by the parameter \code{fieldDump\_every} in the input file. They are written in the file \code{Fields.h5}. The script \code{Diagnostics.py} is capable of opening and plotting these fields together with the particle diagnostics. In fact the procedure is almost the same:
\begin{lstlisting}
Field(results_path, field=None, timesteps=None, slice=None,
       units="code", data_log=False)
\end{lstlisting}
As you can see, \code{Field(...)} works almost the same way as \code{ParticleDiagnostic(...)}, with some exceptions:
\begin{itemize}
\item \code{field} must be one of \code{"Bx\_m"}, \code{"By\_m"}, \code{"Bz\_m"}, \code{"Ex"}, \code{"Ey"}, \code{"Ez"}, \code{"Jx"}, \code{"Jy"}, \code{"Jz"}, \code{"Jx\_\textit{species}"}, \code{"Jy\_\textit{species}"}, \code{"Jz\_\textit{species}"}, \code{"Rho"} or \code{"Rho\_\textit{species}"} where \val{species} is the name of one of the existing species. If you omit the argument \code{field}, the list of available fields will be displayed.\\*
You can also make operations between fields (e.g. \code{"Jx+Jy"}).
\item \code{slice} can only accept three axes: \code{"x"}, \code{"y"}, \code{"z"}. For instance, \code{slice=\{"x":"all"\}}. Note that the slice does \textbf{not} calculate the sum of the axis, but the \textbf{average}.
\end{itemize}

In addition to fields, SMILEI outputs scalar values as a function of time in the file \code{scalars.txt}. You can plot them as well:
\begin{lstlisting}
Scalar(results_path, scalar=None, timesteps=None, units="code", data_log=False)
\end{lstlisting}
The differences are:
\begin{itemize}
\item \code{scalar} must be the name of an available scalar. If you omit this argument, the list of available scalar will be displayed.
\item There is no more \code{slice} argument of course!
\end{itemize}


Please refer to the previous section to understand the other arguments. In addition, all the plotting arguments discussed in the previous section are available for the fields and scalars as well.\\*\\*

All the functions \code{plot()}, \code{set()}, \code{get()} and \code{getData()} are also compatible with \code{Field(...)} and \code{Scalar(...)}.\\*\\*

Last but not least: \code{multiPlot()} can be used to combine particle, field and scalar diagnostics on the same figure.\\*\\*

\textbf{Example:}
\begin{lstlisting}
>>> A = Field("path/to/my/results", field="Ex")
>>> B = ParticleDiagnostic("path/to/my/results", diagNumber=0)
>>> C = ParticleDiagnostic("path/to/my/results", scalar="Ebalance")
>>> multiPlot( A, B, C )
\end{lstlisting}

\clearpage
\section{Tutorial}

If you don't know how to run SMILEI, please refer to the appropriate documentation first. The commands can vary depending on your system and installation. A typical example of the command to run SMILEI is: \code{mpiexec -np 1 smilei mycase.in}\\*

\subsection{Running the test case}

In the \code{benchmarks} directory, we provide a test case \code{tst1d\_6\_particle\_diagnostic.in}. This case is very simple: it consists of a one-dimensional uniform neutral plasma composed by ions and electrons. The electrons all have a drift velocity of $0.05c$.\\*

Run this case using SMILEI and collect the results in a directory of your choice. In this tutorial, we suppose that the results are in the directory \code{tst1d\_6\_particle\_diagnostic}. Make sure this directory contains the input file \code{tst1d\_6\_particle\_diagnostic.in} and the output files \code{ParticleDiagnostic0.h5}, \code{ParticleDiagnostic1.h5}, etc.\\*

An example of the commands you may use from a UNIX \textit{terminal} (or \textit{console}) is
\begin{lstlisting}
$ mkdir tst1d_6_particle_diagnostic
$ cp   benchmarks/tst1d_6_particle_diagnostic.in    tst1d_6_particle_diagnostic
$ cd tst1d_6_particle_diagnostic
$ mpiexec -np 1 smilei tst1d_6_particle_diagnostic.in
$ cd ..
\end{lstlisting}

\subsection{Starting \textit{python} and listing available diagnostics}

From the same terminal, launch \textit{python} using the command
\begin{lstlisting}
$ python -i scripts/Diagnostics.py
\end{lstlisting}

Your are now in the \textit{python} prompt.\\*
Obtain a list of available particle diagnostics using
\begin{lstlisting}
>>> ParticleDiagnostic('tst1d_6_particle_diagnostic')
Printing available particle diagnostics:
----------------------------------------
Diag#0 - density of species # 1 
    Every 4 timesteps, averaging over 2 timesteps
    x from 0.0 to 6.28319 in 100 steps 
    vx from -0.1 to 0.1 in 100 steps 
Diag#1 - density of species # 0 
    Every 4 timesteps, no time-averaging
    x from 0.0 to 6.28319 in 100 steps 
    vx from -0.001 to 0.001 in 100 steps 
Diag#2 - density of species # 1 
    Every 10 timesteps, averaging over 5 timesteps
    ekin from 0.0001 to 0.1 in 100 steps  [ LOG SCALE ] 
\end{lstlisting}

Look at the diagnostic \#0: it is the density of species \#1 (here, electrons) with two
axes: the position $x$ and the velocity $v_x$. In other words, it is the phase-space of electrons.

\subsection{Plot a diagnostic result at $t=0$}
To plot the phase-space in the initial conditions, use 
\begin{lstlisting}
>>> ParticleDiagnostic('tst1d_6_particle_diagnostic', 0, timesteps=0 ).plot()
\end{lstlisting}
A window appears (see Figure \ref{Tuto1}). We can see that the electrons have indeed a drift velocity of $0.05c$.\\*

To obtain the equivalent plot for the ions, use the diagnostic \#1 with the command
\begin{lstlisting}
>>> ParticleDiagnostic('tst1d_6_particle_diagnostic', 1, timesteps=0 ).plot()
\end{lstlisting}
This results in the plot in Figure \ref{Tuto2}. The ions have a zero average velocity.\\*

\begin{figure}[htbp]
\centering
\begin{minipage}[b]{.46\linewidth}
\includegraphics[width=7cm]{ParticleDiagTutorial1}
\caption{Phase-space of electrons at $t=0$.}
\label{Tuto1}
\end{minipage}
\hfill{}
\begin{minipage}[b]{.46\linewidth}
\includegraphics[width=7cm]{ParticleDiagTutorial2}
\caption{Phase-space of ions at $t=0$.}
\label{Tuto2}
\end{minipage}
\end{figure}



\subsection{Plot sections (``slices'') of the array}

The diagnostic \#0 that we plotted in Figure \ref{Tuto1} is the electron phase-space.
Let us say we want to sum over the data that is contained between $x=3$ and 4, and plot the result as a function of $v_x$.
This is achieved by the argument \code{slice}:
\begin{lstlisting}
>>> ParticleDiagnostic('tst1d_6_particle_diagnostic', 0, timesteps=0,
       slice={"x":[3,4]} ).plot()
\end{lstlisting}
The result is shown in Figure \ref{Tuto3}. We can see that the peak is located at $v_x=0.05c$, as we have already found before.\\*

\begin{figure}[htbp]
\centering
\begin{minipage}[b]{.46\linewidth}
\includegraphics[width=7cm]{ParticleDiagTutorial3}
\caption{$v_x$ distribution of electrons contained between $x=3$ and 4, at $t=0$.}
\label{Tuto3}
\end{minipage}
\hfill{}
\begin{minipage}[b]{.46\linewidth}
\includegraphics[width=7cm]{ParticleDiagTutorial4}
\caption{$x$ distribution of electrons contained between $v_x=-0.1$ and $0.1$, at $t=0$.}
\label{Tuto4}
\end{minipage}
\end{figure}

Now, let us do the slice on $v_x$ instead of $x$.
\begin{lstlisting}
>>> ParticleDiagnostic('tst1d_6_particle_diagnostic', 0, timesteps=0,
       slice={"vx":"all"} ).plot(data_min=0, data_max=11)
\end{lstlisting}
By choosing \code{"all"} in the argument \code{slice}, all the velocities $v_x$ are sliced.
In our case, as our diagnostic goes from $v_x=-0.1$ to $0.1$, these limits are used.\\*
Note that parameters \code{data\_min} and \code{data\_max} are used to have a nicer plot.\\*
The result is shown in Figure \ref{Tuto4}. We obtain a constant density of 10$n_c$, which is what was chosen in the input file.

\subsection{Make animated plots}
To have an animation of the electron phase-space with time, you have to remove the \code{timesteps} argument:
\begin{lstlisting}
>>> ParticleDiagnostic('tst1d_6_particle_diagnostic', 0 ).plot()
\end{lstlisting}
You will see the electron velocity oscillate from $0.05c$ to $-0.05c$. This is due to the fact that we are simulating a plasma wave with infinite wavelength.\\*

Note that all the available timesteps are animated. If you want to only animate between timesteps 20 and 60, use
\begin{lstlisting}
>>> ParticleDiagnostic('tst1d_6_particle_diagnostic', 0, timesteps=[20,60] ).plot()
\end{lstlisting}


\subsection{Make multiple plots on the same figure}

Use the following commands to have the animation with both electrons and ions on the same figure:
\begin{lstlisting}
>>> A = ParticleDiagnostic('tst1d_6_particle_diagnostic', 0 )
>>> B = ParticleDiagnostic('tst1d_6_particle_diagnostic', 1 )
>>> multiPlot(A, B, shape=[1,2])
\end{lstlisting}
A snapshot of this double plot is given in Figure \ref{Tuto5}.\\*\\*

If the two plots are 1D, and are both of the same type, then they will automatically be plotted on the same axes.
\begin{lstlisting}
>>> A = ParticleDiagnostic('tst1d_6_particle_diagnostic',0,slice={"x":"all"})
>>> B = ParticleDiagnostic('tst1d_6_particle_diagnostic',1,slice={"x":"all"})
>>> multiPlot(A, B)
\end{lstlisting}
This is shown in Figure \ref{Tuto6} where you can see the two curves in blue and green.\\*\\*

\begin{figure}[htbp]
\centering
\begin{minipage}[b]{.48\linewidth}
\includegraphics[width=10cm]{ParticleDiagTutorial5}
\caption{Two plots on the same figure.}
\label{Tuto5}
\end{minipage}
\hfill{}
\begin{minipage}[b]{.38\linewidth}
\includegraphics[width=6cm]{ParticleDiagTutorial6}
\caption{Two curves in the same axes.}
\label{Tuto6}
\end{minipage}
\end{figure}


\subsection{Make a plot as a function of time}
If you have sliced all the axes, then you obtain a 0-dimensional array (a scalar). In this case, the plots are automatically done 
as a function of time (they are not animated).\\*
In our case, use
\begin{lstlisting}
>>> A=ParticleDiagnostic('tst1d_6_particle_diagnostic', 3, slice={"ekin":"all"})
>>> B=ParticleDiagnostic('tst1d_6_particle_diagnostic', 3, slice={"ekin":[0,0.001]})
>>> multiPlot(A,B)
\end{lstlisting}

\begin{wrapfigure}{r}{80mm}
  \centering
  \includegraphics[width=6.5cm]{ParticleDiagTutorial7}
  \caption{Blue: total density \emph{vs} time. Green: density of slow electrons \emph{vs} time.}
  \label{Tuto7}
\end{wrapfigure}
The diagnostic that we employ here (\#3) is the energy spectrum of electrons: the axis is along \code{ekin} which is the kinetic energy. In the first line of the code above, we are using a slice \code{"ekin":"all"}. Consequently, all the electrons, with all energies, will be summed, thus obtaining a scalar value equal to the total plasma density. In the second line of code, we are using \code{"ekin":[0,0.001]}, which means that only the electrons below 0.511 keV are considered.\\*
Both these quantities \code{A} and \code{B} are scalars, not arrays: they will be plotted as a function of time. This is shown in Figure \ref{Tuto7} where you can see \code{A} in blue and \code{B} in green. \code{A} represents all the electrons, and indeed, their density is constant. \code{B} represents only the slower electrons, and their number varies in time because, as we have seen before, all electrons oscillate and they do not have a constant energy. This appears on the green curve as an oscillating density.



\subsection{Make an operation between diagnostics}
Let us consider again the diagnostic \#0, which is the density of electrons as a function of $x$ an $v_x$. Diagnostic \#2 is very similar to \#0 as it has the same axes $x$ and $v_x$, but it has \code{ouput=px\_density} instead of \code{ouput=density}. Consequently, if we divide \#2 by \#0, we will obtain the average value of $p_x$ as a function of $x$ an $v_x$. To do this operation, we need to indicate \code{"\#2/\#0"} instead of the diagnostic number:
\begin{lstlisting}
>>> ParticleDiagnostic('tst1d_6_particle_diagnostic', "#2/#0").plot()
\end{lstlisting}
We obtain the plot of Figure \ref{Tuto8}, which is actually not very helpful because $\left<p_x\right>$ varies with $v_x$. To have something nicer, let us slice all axes with \code{slice=\{"x":"all","vx":"all"\}}.
\begin{lstlisting}
>>> ParticleDiagnostic('tst1d_6_particle_diagnostic', "#2/#0",
          slice={"x":"all","vx":"all"}).plot()
\end{lstlisting}
We obtain Figure \ref{Tuto9} which nicely shows the average $p_x$ as a function of time. This value oscillates, as we have seen previously.


\begin{figure}[htbp]
\centering
\begin{minipage}[b]{.48\linewidth}
\includegraphics[width=7.5cm]{ParticleDiagTutorial8}
\caption{$\left<p_x\right>$ as a function of $x$ and $v_x$.}
\label{Tuto8}
\end{minipage}
\hfill{}
\begin{minipage}[b]{.48\linewidth}
\includegraphics[width=7.3cm]{ParticleDiagTutorial9}
\caption{$\left<p_x\right>$ as a function of time.}
\label{Tuto9}
\end{minipage}
\end{figure}




\end{document}


