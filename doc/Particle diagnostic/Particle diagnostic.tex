\documentclass[11pt]{article}
\usepackage{graphicx}
\usepackage[margin=2cm]{geometry}
\usepackage{fancyvrb,xcolor}

\setlength\parindent{0pt}

\newcommand{\code}[1]{\colorbox{yellow!15}{\ttfamily #1}}
\newcommand{\val}[1]{{\ttfamily \textit{#1}}}

\usepackage{listings}
\lstset{backgroundcolor=\color{yellow!15}, basicstyle=\ttfamily, columns=fullflexible, keepspaces=true,escapeinside={�}{�}}

\begin{document}

\title{The particle diagnostic in SMILEI}
\author{F. P\'erez}
\date{March 19, 2015}
\maketitle

The \textit{particle diagnostic} collects data from the macro-particles and processes them during runtime.
It does \underline{not} provide information on individual particles: instead, it produces \underline{averaged quantities}
like the particle density, currents, etc.
\\*
\\*
The data may be collected from one or several particle species.
\\*
\\*
The data is discretized inside a ``grid'' chosen by the user. This grid may be of any dimension.
\\*
Examples:
\begin{itemize}
\item 1-dimensional grid along the position $x$ (gives density variation along $x$)
\item 2-dimensional grid along positions $x$ and $y$ (gives density map)
\item 1-dimensional grid along the velocity $v_x$ (gives the velocity distribution)
\item 2-dimensional grid along position $x$ and momentum $p_x$ (gives the phase-space)
\item 1-dimensional grid along the kinetic energy $E_\mathrm{kin}$ (gives the energy distribution)
\item 3-dimensional grid along $x$, $y$ and $E_\mathrm{kin}$ (gives the density map for several energies)
\item 1-dimensional grid along the charge $Z^\star$ (gives the charge distribution)
\end{itemize}
Each dimension of the grid is called ``axis".
\\*
\\*
The user may choose to average the data over several time-steps.


\clearpage

\section{How to add a particle diagnostic in the input file\\* \\*}
In the input file, you can add a particle diagnostic with the following syntax.

\begin{lstlisting}
diagnostic particles
	output = �\textit{output}�
	every = �\textit{every}�
	time_average = �\textit{time\_average}�
	species = �\textit{species1 species2 ...}�
	axis = �\textit{type min max nsteps}� [logscale] [edge_inclusive]
	axis = �\textit{type min max nsteps}� [logscale] [edge_inclusive]
	...
end
\end{lstlisting}

\begin{itemize}
\item \val{output} must be set to \code{density}, \code{charge\_density}, \code{current\_density\_x}, \code{current\_density\_y}, \code{current\_density\_z}.
	It determines the data that is summed in each cell of the grid.
	In the case of \code{density}, the \textit{weights} are summed.
	In the case of \code{charge\_density}, the \textit{weights}$\times$\textit{charge} are summed.
	In the case of \code{current\_density\_x}, the \textit{weights}$\times$\textit{charge}$\times v_x$ are summed (etc.)\\*
\item \val{every} must be a positive integer. It is the number of time-steps between each output.\\*
\item \val{time\_average} must be a positive integer. It is the number of time-steps during which the data is averaged before output.\\*
\item \val{species1 species2 ...}  must be one or several species.
	Species are recognized by their parameter \code{species\_type} that is included
	in each group \code{species ... end}\\*
\item \code{axis} is an argument that describes one axis of the grid.\\
	Syntax: \code{axis = \textit{type min max nsteps} [logscale] [edge\_inclusive]}\\
	\val{type} can be \code{x}, \code{y}, \code{z}, \code{px}, \code{py}, \code{pz}, \code{p}, 
	 \code{gamma}, \code{ekin}, \code{vx}, \code{vy}, \code{vz}, \code{v} or \code{charge}.\\
	The axis is discretized for \val{type} from \val{min} to \val{max} in \val{nsteps} bins.\\
	The optional keyword \code{logscale} sets the axis scale to logarithmic instead of linear.\\
	The optional keyword \code{edge\_inclusive} includes the particles outside the range [\val{min},  \val{max}]
	into the extrema bins.\\*\\*
\end{itemize}
There may be as many \code{axis} arguments as wanted in one \code{diagnostic particles} block.
\\*\\*\\*
There may be as many \code{diagnostic particles} blocks as wanted in the input file.

\clearpage
In the input file, you may add the following block as a reminder.

\begin{lstlisting}
# DIAGNOSTICS ON PARTICLES - project the particles on a N-D arbitrary grid
# ------------------------------------------------------------------------
# output = density, charge_density or current_density_[xyz]
#              => parameter that describes what quantity is obtained 
# every        => integer > 0 : number of time-steps between each output
# time_average => integer > 0 : number of time-steps to average
# species      => list of one or several species whose data will be used
# axis   = _type_ _min_ _max_ _nsteps_ [logscale] [edge_inclusive]
#              => _type_ can be one of the following:
#                  x, y, z, px, py, pz, p, gamma, ekin, vx, vy, vz, v or charge
#              => the data is discretized for _type_ between _min_ and
#                  _max_, in _nsteps_ bins
#              => the optional [logscale] sets the scale to logarithmic
#              => the optional [edge_inclusive] forces the particles
#                  outside (_min_,_max_) to be counted in the extrema bins
#   example : axis = x 0 1 30
#   example : axis = px -1 1 100 
# >>>> MANY AXES CAN BE ADDED IN A SINGLE DIAGNOSTIC <<<<
\end{lstlisting}


\clearpage
\section{Examples}

Variation of the density of species \code{electron1} from $x=0$ to 1, every 5 time-steps, without time-averaging.
\begin{lstlisting}
diagnostic particles
	output = density
	every = 5
	time_average = 1
	species = electron1
	axis = x    0    1    30
end
\end{lstlisting}

Density map from $x=0$ to 1, $y=0$ to 1.
\begin{lstlisting}
diagnostic particles
	output = density
	every = 5
	time_average = 1
	species = electron1
	axis = x    0    1    30
	axis = y    0    1    30
end
\end{lstlisting}

Velocity distribution from $v_x = -0.1$ to $0.1$.
\begin{lstlisting}
diagnostic particles
	output = density
	every = 5
	time_average = 1
	species = electron1
	axis = vx    -0.1    0.1    100
end
\end{lstlisting}

Phase space from $x=0$ to 1 and from $px=-1$ to 1.
\begin{lstlisting}
diagnostic particles
	output = density
	every = 5
	time_average = 1
	species = electron1
	axis = x    0    1    30
	axis = px    -1    1    100
end
\end{lstlisting}

Energy distribution from 0.01 to 1 MeV in logarithmic scale.
Note that the input units are $m_ec^2 \sim 0.5$ MeV.
\begin{lstlisting}
diagnostic particles
	output = density
	every = 5
	time_average = 1
	species = electron1
	axis = ekin    0.02    2   100  logscale
end
\end{lstlisting}
\clearpage

$x$-$y$ density maps for three bands of energy: $[0,1]$, $[1,2]$, $[2,\infty]$.
Note the use of \code{edge\_inclusive} to reach energies up to $\infty$.
\begin{lstlisting}
diagnostic particles
	output = density
	every = 5
	time_average = 1
	species = electron1
	axis = x    0    1    30
	axis = y    0    1    30
	axis = ekin    0   6   3  edge_inclusive
end
\end{lstlisting}

Charge distribution from $Z^\star =0$ to 10.
\begin{lstlisting}
diagnostic particles
	output = density
	every = 5
	time_average = 1
	species = electron1
	axis = charge    -0.5   10.5   11
end
\end{lstlisting}

\clearpage
\section{How to view and post-process a particle diagnostic}

Each diagnostic produces one file like \textit{ParticleDiagnostic0.h5}, \textit{ParticleDiagnostic1.h5}, etc. They
are numbered the same way as the order of appearance in the input file.
\\*

A python script \textit{ParticleDiagnostic.py} is provided to view or extract data from these files.
To run this script, you will need \textit{python2.7} with the following packages: numpy, matplotlib, pylab, h5py.
\\*

First, run python and include the script:
\begin{lstlisting}
python -i ParticleDiagnostic.py
\end{lstlisting}

(Alternately, you can include this file into your own script using, for example, \\
\code{execfile("scripts/ParticleDiagnostic.py")})
\\*

In python, you can run \code{ParticleDiagnostic(...)} to extract the data and optionally plot it.
\\*

\underline{Syntax of the function \textit{ParticleDiagnostic}}:
\begin{lstlisting}
ParticleDiagnostic(results_path, diagNumber=None, timesteps=None, slice=None,
                    units="code", data_log=False, data_min=None, data_max=None,
                    xmin=None, xmax=None, ymin=None, ymax=None,
                    figure=None)
\end{lstlisting}

\begin{itemize}
\item \code{results\_path} = \code{\_string\_}\\*
	Path to the directory where the outputs are stored.\\*
	(Also, this has to contain one and only one input file *.in)

\item \code{diagNumber} = \code{\_int\_} (optional)\\*
	Number of the diagnostic. The first diagnostic has number 0.\\*
	If not given, then a list of available diagnostics is printed.

\item \code{timesteps} =  \code{\_int\_ }          (optional)\\*
	\code{timesteps} = \code{[\_int\_, \_int\_]}   (optional)\\*
	If omitted, all timesteps are used.\\*
	If one number  given, the nearest timestep available is used.\\*
	If two numbers given, all the timesteps in between are used.

\item \code{slice} = \code{\{ \val{axis} : "all", ... \}}                 (optional)\\*
	\code{slice} = \code{\{ \val{axis} : \_double\_, ... \}}              (optional)\\*
	\code{slice} = \code{\{ \val{axis} : [\_double\_, \_double\_], ... \}}  (optional)\\*
	This parameter is used to reduce the number of dimensions of the array.\\*
	\val{axis} must be \code{"x"}, \code{"y"}, \code{"z"}, \code{"px"}, \code{"py"}, \code{"pz"}, \code{"p"},
	 \code{"gamma"}, \code{"ekin"}, \code{"vx"}, \code{"vy"}, \code{"vz"}, \code{"v"} or \code{"charge"}.\\*
	 Any axis of the same name will be removed with the following technique:\\*
	- If the value is \code{"all"}, then a sum is performed over all the axis.\\*
	- If the value is \code{\_double\_}, then only the bin closest to the value is kept.\\*
	- If the value is \code{[\_double\_,\_double\_]}, then a sum is performed between the two values.\\*
	Example: \code{\{"x":[4,5]\}} will sum all the data for x in the range [4,5].

\item \code{units} = \code{"nice"}    (optional)\\*
	If \code{"nice"} is chosen, then units are converted into usual units.\\*
	Distances in microns, density in cm$^{-3}$, energy in MeV.

\item \code{data\_log} = \code{True} or (\code{False})       (optional)\\*
	If True, then $\log10$ is applied to the output array before plotting.

\item \code{data\_min} = \code{\_double\_}    (optional)\\*
	\code{data\_max} = \code{\_double\_}    (optional)\\*
	If present, output is rescaled before plotting.

\item \code{xmin} = \code{\_double\_}    (optional)\\*
	\code{xmax} = \code{\_double\_}    (optional)\\*
	\code{ymin} = \code{\_double\_}    (optional)\\*
	\code{ymax} = \code{\_double\_}    (optional)\\*
	If present, axes are rescaled before plotting.

\item \code{figure} = \code{None}        (default)\\*
	\code{figure} = \code{\_int\_}       (optional)\\*
	Choses the figure number that is passed to matplotlib.\\*
	If absent or None, returns the first data without plotting.\\*
\end{itemize}


\underline{Output of the function \textit{ParticleDiagnostic}}

If \code{figure} is a number, then the function does not return anything. Instead, it displays the results in a graph.
\\*

If \code{figure} is not provided, or \code{figure=None}, then the function returns the results as a python dictionary.
The dictionary is of the form \code{\{"data":\textit{data\_array}, \textit{axis1}:\textit{axis1\_array}, \textit{axis2}:\textit{axis2\_array}, ...\}}
where:
\begin{itemize}
\item \val{data\_array} is the final array containing the data.
\item \val{axis1} and \val{axis2} are the names of the axes (for example \code{"x"} or \code{"px"}).
\item \val{axis1\_array} and \val{axis2\_array} are the locations of the axes bins.\\*\\*
\end{itemize}

\underline{Examples of usage from the python prompt:}
\begin{lstlisting}
>>> ParticleDiagnostic("path/to/my/results", diagNumber=1, slice={"y":"all"},
   units="nice",data_min=0, data_max=3e14, figure=1)
\end{lstlisting}
This will take the diagnostic \#1, and sum the grid over all the $y$ axis. Then it will plot the resulting array in figure 1 from 0 to 3e14.
\\*

\begin{lstlisting}
>>> result = ParticleDiagnostic("path/to/my/results", diagNumber=3,
    slice={"ekin":[1,10]}, timesteps=1000)
\end{lstlisting}
This will take the diagnostic \#3 for the timestep nearest to 1000, and sum for all energies between 1 and 10.
The results are stored in the variable \code{result}.\\*
The data can be accessed with \code{result["data"]}.\\*
If one of the axes is \val{x}, you can access the locations of the bins with \code{result["x"]}.



\end{document}


